\documentclass[margin,line]{res}

\usepackage{amssymb, amsthm, amsmath, amsfonts}

\oddsidemargin -.5in
\evensidemargin -.5in
\textwidth=6.0in
\itemsep=0in
\parsep=0in

\newenvironment{list1}{
  \begin{list}{\ding{113}}{%
      \setlength{\itemsep}{0in}
      \setlength{\parsep}{0in} \setlength{\parskip}{0in}
      \setlength{\topsep}{0in} \setlength{\partopsep}{0in} 
      \setlength{\leftmargin}{0.17in}}}{\end{list}}
\newenvironment{list2}{
  \begin{list}{$\bullet$}{%
      \setlength{\itemsep}{0in}
      \setlength{\parsep}{0in} \setlength{\parskip}{0in}
      \setlength{\topsep}{0in} \setlength{\partopsep}{0in} 
      \setlength{\leftmargin}{0.2in}}}{\end{list}}


\begin{document}

\name{Matthew M. McCormick \vspace*{.1in}}

\begin{resume}
\section{\sc Contact Information}
\vspace{.05in}
\begin{tabular}{@{}p{3in}p{4in}}
Wisconsin Institutes for Medical Research & {\it Voice:}  (763) 670-6479 \\            
1111 Highland Avenue, Room 1005           & {\it E-mail:}  matt@mmmccormick.com \\       
Madison, WI 53705 USA                     & {\it WWW:} http://mmmccormick.com\\     
\end{tabular}


\section{\sc Objective}
Medical image analysis research with open source software.  Diagnostic
ultrasound preferred, but also interested in other imaging modalities.

\section{\sc Education}
{\bf University of Wisconsin-Madison} \\
%{\em Department of Statistics} 
\vspace*{-.1in}
\begin{list1}
\item[] Ph.D. Biomedical Engineering, May 2011
\begin{list2}
\vspace*{.05in}
\item Dissertation Topic:  ``Carotid Plaque Characterization with Medical
Ultrasound''
%   Spatial Data and Regression Problems'' 
%\item Dissertation Topic:  ``Hierarchical Models for Multiple Ratings
%  in Performance-Based\\ \hspace*{1.23in} Student Assessments.'' 
\item GPA: 3.70/4.0
\end{list2}
\vspace*{.05in}
% \item[] M.S., Statistics,  May 2000
\item[] M.S. in Biomedical Engineering, May 2008
\end{list1}

{\bf Marquette University}, Milwaukee, Wisconsin USA\\
%{\em Department of Mathematics and Statistics} 
\vspace*{-.1in}
\begin{list1}
\item[] B.S., Biomedical Engineering,  May, 2005

\begin{list2}
 \vspace*{.05in}
 \item Minor: Mathematics, Biology
 \item Marquette University Honors Program
 \item GPA: 3.88/4.0
\end{list2}
\end{list1}

{\bf St. Mary's University of Minnesota}, Winona, Minnesota USA\\
\vspace*{-.1in}
\begin{list1}
\item[] 1999-2001
\end{list1}


{\bf Minnesota Academy of Mathematics and Science}, Winona, Minnesota USA\\
\vspace*{-.1in}
\begin{list1}
\item[] Graduated May, 2001
\end{list1}

{\bf Cotter High School}, Winona, Minnesota USA\\
\vspace*{-.1in}
\begin{list1}
\item[] Graduated May, 2001
\end{list1}

\section{\sc Work\\ Experience}

{\bf University of Wisconsin-Madison}, USA

\vspace{-.3cm}
{\em Research Assistant} \hfill {\bf June 2005 - Present} \\
Research diagnostic medical ultrasound stiffness imaging methods for non-invasive assessment of the carotid artery.

\vspace{-.3cm}
{\em Teaching Assistant} \hfill {\bf January 2009 - May 2009} \\
Prepared, taught, and graded laboratories for the Diagnostic Ultrasound Physics
course.

{\bf Neuromotor Control Laboratory} Marquette University, Milwaukee, WI USA

\vspace{-.3cm}
{\em Research Assistant} \hfill {\bf May 2004 - May 2005} \\
Electronic/mechanical hardware development for MRI compatible wrist robot, data processing and analysis for understanding of nervous system use of sensory information,  adaptation,  and control of the skeletal muscle system.

{\bf Boston Scientific Corporation} Maple Grove, MN USA

\vspace{-.3cm}
{\em Research and Design Intern} \hfill {\bf June 2003 - August 2003} \\
Research and development on peripheral vascular self-expanding Nitinol stents.

{\bf Educational Opportunity Program} Marquette University, Milwaukee, WI USA

\vspace{-.3cm}
{\em Tutor} \hfill {\bf August 2002 - May 2005} \\
Tutor college students in courses such as biology, chemistry, and calculus.

{\bf Minnesota Association for Human Genetics} University of Minnesota, Minneapolis, MN USA

\vspace{-.3cm}
{\em Research Intern} \hfill {\bf May 2000 - June 2000} \\
Perform genetic sequence analysis on the tyrosinase gene of individuals with albinism to probe for mutations.

\section{\sc Publications and Manuscripts}

McCormick, M and Varghese, T.  An Approach to Unbiased Subsample Interpolation
For Motion Tracking.  Ultrasonics.  In Review. 2011.

McCormick, M, Madsen, E; Deaner, M; Varghese, T.  Absolute Backscatter
Measurement of Tissue-Mimicking Phantoms in the 5-45 MHz Frequency Range.  Journal of the
Acoustical Society of America.  In Review.  2011.

McCormick, M, Rubert, N and Varghese, T.  Bayesian Regularization Applied to
Ultrasound Strain Imaging.  IEEE Transactions on Biomedical Engineering.
In Press.  2011.

Perrot-Audet, A, McCormick, M, Gelas, A, Rannou, N, Souhait, L, Mosaliganti, K,
and Megason, S.  A Lightweight Image Comparison Library.  Kitware's The Source.
Issue 16, January 2011.
\vspace*{-.25in}
URL: \begin{verbatim}http://www.kitware.com/products/html/ALightweightImageComparisonLibrary.html\end{verbatim}.
\vspace*{-.35in}

McCormick, M.  Ultrasound and ITKv4.  Kitware's The Source.  Issue 16, January
2011.
\vspace*{-.25in}
URL: \begin{verbatim}http://www.kitware.com/products/html/UltrasoundAndITKv4.html\end{verbatim}.
\vspace*{-.35in}

Madsen, E; Frank, G; McCormick, M; Deaner, M.  Anechoic Sphere Phantoms for
Estimating 3-D Resolution of Very High Frequency Ultrasound Scanners.
IEEE Transactions on Ultrasonics, Ferroelectrics, and Frequency Control. 57
(10):2284-2292. 2010.

McCormick, M.  Higher Order Accurate Derivative and Gradient Calculation in ITK.
Insight Journal.  2010 July-December.
\vspace*{-.25in}
URL: \begin{verbatim}http://hdl.handle.net/10380/3231\end{verbatim}.
\vspace*{-.35in}

McCormick, M.  Visual Debugging of ITK.  Kitware's The Source.  Issue 13, April
2010.

McCormick, M.  An Open Source, Fast Ultrasound B-Mode Implementation for
Commodity Hardware.  Insight Journal.  2010 January-June.
\vspace*{-.25in}
URL: \begin{verbatim}http://hdl.handle.net/10380/3159\end{verbatim}.
\vspace*{-.35in}

Shi, H; Varghese, T; Mitchell, C; McCormick, M; Dempsey, RJ; Kliewer, MA.
In-vivo Attenuation and Equivalent Scatter size parameters for Atherosclerotic
Carotid Plaque: Preliminary Results.  Ultrasonics 49 (8):779-785.  2009.

Shi, H; Mitchell, CC; McCormick, M; Kliewer, MA; Dempsey, RJ; Varghese, T.
Preliminary in vivo atherosclerotic carotid plaque characterization
using the accumulated axial strain and relative lateral shift strain
indices.  Phys Med Biol. 53 (22):6377-94. 2008. PMID:
18941278

\section{\sc Conference Presentations}

McCormick, M and Varghese, T.  Reduction of Reverberation Artifacts in Carotid
Strain Images Using Bayesian Regularization.  International Conference on the
Ultrasonic Measurement and Imaging of Tissue Elasticity.  Oct 16, 2010.
Snowbird, Utah.

McCormick, M and Varghese, T.  Subsample Displacement Interpolation Using
Windowed-Sinc Reconstruction with Numerical Optimization.  International
Conference on the Ultrasonic Measurement and Imaging of Tissue Elasticity.  Oct
16, 2010.  Snowbird, Utah.

McCormick, M and Varghese, T.  Open Technologies Applied to a Non-standard Medical
Image Format for Innovative Research.  MathBio2: IMAGE.  November 2009.  Madison, WI.

McCormick, M; Varghese, T; Dempsey, RJ; Zagzebski, J; Madsen, E.  High Frequency Ultrasonic Characterization of Excised Atherosclerotic
Carotid Plaque.  Ultrasonic Imaging and Tissue Characterization
Symposium.  June 2009.  Arlington, VA.

Madsen, E;  McCormick, M;  Frank, G.  Phantoms for Assessing
Intravascular (IVUS) Ultrasound Scanners.  American Institutes in
Ultrasound and Medicine Conference.  April 2009.  New York, NY.

McCormick, M; Shi, H; Mitchell C; Kliewer M; Dempsey R; Varghese T.   Mechanical
Viscoelastic Variations of \textit{in vivo} Carotid Atheromas using External
Ultrasound.  Fifth International Conference on the Ultrasonic Measurement and
Imaging of Tissue Elasticity.  Oct 8, 2006.  Snowbird, Utah USA.

%\pagebreak[4]


% 
% 
% Paciorek, C.J., J.S. Risbey, V. Ventura, and R.D.Rosen.  2001.  Changes in Northern Hemisphere winter storm activity (1949-1999) based
% on a comparison of cyclone indices.  8th International Meeting on
% Statistical Climatology, Luneberg, Germany, March, 2001.
% 
% Paciorek, C.J. and R. Rosenfeld.  2000.  Minimum classification error
% training in exponential language models.  2000 Spring Transcription
% Workshop, College Park, Maryland.
% \vspace*{-.25in}  
% \begin{verbatim}http://www.nist.gov/speech/publications/tw00/html/abstract.htm#cp1-50\end{verbatim}
% 

%\pagebreak[4]
%
\section{\sc Computer Skills} 
\begin{list2}
\item Languages:  C++, Python, Matlab, and Bash.
\item Operating Systems:  Linux, Windows.\\ 
\end{list2}
\vspace*{-.15in}

Patches submitted to and accepted at:
\begin{list2}
\item Awesome Window Manager
  \begin{verbatim}http://awesome.naquadah.org/\end{verbatim}
\item Bioimage Suite.  Medical image processing and visualization.
  \begin{verbatim}http://www.bioimagesuite.org/\end{verbatim}
\item cgit. A fast web-interface for git repositories. \begin{verbatim}http://hjemli.net/git/cgit/about/\end{verbatim}
\item CMake.  C and C++ configuration tool.
  \begin{verbatim}http://www.cmake.org/\end{verbatim}
\item gccxml.  XML output for GCC.
  \begin{verbatim}http://www.gccxml.org/\end{verbatim}
\item Gentoo.  Linux distribution.
  \begin{verbatim}http://www.gentoo.org/\end{verbatim}
\item InsightToolkit.  Insight Segmentation and Registration Toolkit.  \textit{Developer status}.
  \begin{verbatim}http://itk.org/\end{verbatim}
\item Pyclewn.  Pyclewn allows using vim as a front end to a debugger.
  \begin{verbatim}http://pyclewn.sourceforge.net/\end{verbatim}
\item QGoImageCompare.  QGoImageCompare is a library aimed at simple comparison
  of images.
  \begin{verbatim}https://github.com/gofigure2/QGoImageCompare/\end{verbatim}
\item usimagtool.  Medical ultrasound image processing tool.
  \begin{verbatim}http://www.lpi.tel.uva.es/usimag/en/ContenidoEn.php?IdContenido=6/\end{verbatim}
\item veusz.  Veusz is a scientific plotting and graphing package written in
  Python.  \begin{verbatim}http://home.gna.org/veusz/\end{verbatim}
\item vistrails.  VisTrails is an open-source scientific workflow and provenance
  management system that provides support for data exploration and visualization.
  \begin{verbatim}http://vistrails.org/\end{verbatim}
\end{list2}

\section{\sc Awards and Activities}
InSCIght.  The Scientific Computing Podcast.  
\begin{list2}
\item Moderator/Panelist.
\item \verb#http://inscight.org/#
\end{list2}

IEEE Student Member.

UW-Madison The Hacker Within.  A peer-teaching group whose purpose is to provide
non-computer scientists with the practical skills required to perform research.
\begin{list2}
\item Organizing member of the 2011 Software Carpentry Bootcamp.
\item Arranged university-sponsored guest lecture of Dr. John D. Hunter from Chicago.
\item Organizing member of the 2010 Python Bootcamp.
\item Presentations on CMake and creating custom pretty-printers in GDB.
\item Representation at PyCon 2010.
\end{list2}

2009 Department of Medical Physics Outstanding Teacher Award.
\begin{list2}
\item  Nomination by students.
\end{list2}

Clinical Neuroengineering Training Program, University of Wisconsin-Madison,
2008-2009.

Marquette University Honors Program.

Alpha Eta Mu Beta, National Biomedical Engineering Honor Society.
\begin{list2}
 \item Local Chapter Secretary, 2003 - 2004
 \item President, 2004 - 2005
\end{list2}

Pi Mu Epsilon- National Mathematics Honor Society.

Marquette University Concert, Jazz, Doc C's Combo, Orchestral, and Pep Bands.

Biomedical Engineering Society, BMES.

Marquette 2002 Engineering Outstanding Sophomore.
\begin{list2}
\item Graduated with High Scholastic Honors
\end{list2}

Rehabilitation Engineering Research Centers on Accessible Medical
Instrumentation.
\begin{list2}
 \item First Place in category, Second Place overall for project on Accessible Syringe Dosing 2004-2005
 \item \begin{verbatim}http://www.eng.mu.edu/wintersj/b18/\end{verbatim}
\end{list2}
%\vspace*{-2.5mm}
%NSF Vertical Integration of Research and Education in Statistics and
%Mathematical Sciences\\ (VIGRE) teaching fellowship.
%


%\pagebreak[4]



\end{resume}
\end{document}




